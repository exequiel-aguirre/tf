\documentclass[12pt]{article}
\usepackage[spanish]{babel}
\usepackage{amsmath}
\usepackage{amssymb}
\usepackage{color}
\usepackage[utf8]{inputenc}
\usepackage{pslatex}
\usepackage{theorem}
\newtheorem{theorem}{Teorema}
\newtheorem{lemma}{Lema}
\newtheorem{proposition}{Proposición}
\newtheorem{corollary}{Corolario}
\newtheorem{example}{Ejemplo}
\newtheorem{note}{Nota}
\newtheorem{definition}{Definición}
\newenvironment{proof}{\paragraph{Demostración:}}{\hfill$\square$}

%\author{Exequiel Aguirre}
\title{Subgrupos discretos del grupo de Heisenberg }
\date{}

\makeindex
% document start
\begin{document}
\maketitle

\begin{abstract}
La teoría de series de Fourier es la versión clásica del análisis armónico en el toro,
$\mathbb{R}^n /\mathbb{Z}^n$, donde $\mathbb{Z}^n$ denota el subgrupo discreto de $\mathbb{R}^n$ de
las n-uplas de números enteros.

Si en lugar de $\mathbb{R}^n$ , consideramos el \index{grupo de Heisenberg} 2n+1 dimensional $\mathbb{H}_n$, un tema
de interés actual es desarrollar el análisis armónico asociado a $\mathbb{H}_n/\Gamma$, donde $\Gamma$ 
denota el subgrupo discreto de $\mathbb{H}_n$. Ver \cite{Th} 

El objetivo de este trabajo es estudiar la descripción hecha por R. Tolimieri \cite{To}.
\end{abstract}

\textbf{Palabras Clave:} Heisenberg,Lattice,grupo,discretos


\textbf{Código de clasificación:}

\clearpage
\tableofcontents
\clearpage

\section{Notación}
\begin{tabular}{ l l }
  $GL(2n,\mathbb{R})$ & Grupo simpléctico\\
  $\mathbb{H}_n$ & Grupo de Heisenberg \\
  $\mathbb{A}_n$ & Grupo de automorfismos de $\mathbb{H}_n$\\
  $\psi$ & Forma bilineal, alternante, no degenerada  en $\mathbb{R}^{2n}$ \\
  $h=(x,y,t)=(v,t)$ & $h \in \mathbb{H}_n$, $x,y \in \mathbb{R}^n,v \in \mathbb{R}^{2n}$, $t \in \mathbb{R}$ \\
  $\Gamma$ & Subgrupo discreto $\mathbb{H}_n$ \\
  ${[}\Gamma,\Gamma{]}$ & Subgrupo conmutador de $\Gamma$ \\
  $L$ & Conjunto de todos los lattices de $\mathbb{H}_n$ \\
  $Z$ & Centro de $\mathbb{H}_n$ \\
  $<v_1,...,v_n>$ & $\mathbb{Z}$-módulo generado por los elementos $v_1,...,v_n$\\
  $\beta(\Gamma)$ & Real positivo unívocamente determinado,\\
  $\Gamma(l)$ & Grupo generado por $(e_1,0)...(e_n,0),(l_1 e_{n+1},0),...,(l_n e_{2n},0)$\\  
  & tal que $\Gamma \cap Z =\{(0,\beta(\Gamma) m): m\in \mathbb{Z} \}$ \\
\end{tabular}


\section{Preliminares}
\subsection{Formas bilineales}
En esta sección describimos la forma canónica de una forma bilineal,
alternante, no degenerada sobre un espacio vectorial real $V$ de dimensión finita. 
Recordamos que una tal forma bilineal $\psi :V\times V\rightarrow R$
es alternante si $\psi(v,v)=0$ $\forall v \in V$  y se dice no degenerada si 
$\psi(v,v')=0$ $\forall v' \in V \Rightarrow v = 0$  y 
$\psi(v,v')=0$ $\forall v \in V \Rightarrow v' = 0$

\begin{definition} 
 Un plano hiperbólico para una forma alternante es un espacio vectorial de dimensión 2 no degenerado,
  i.e. $\psi|_P$ es no degenerada. 
\end{definition}

\begin{definition}   
 Decimos que un espacio vectorial V es hiperbólico si se puede escribir como suma ortogonal (directa ) de
 planos hiperbólicos
\end{definition}

\begin{theorem} \label{Lang-8.1}
 Sea V, un espacio vectorial real de dimensión finita, $\psi$ una forma bilineal, alternante, no degenerada, 
 $\psi:VxV\mapsto \mathbb{R}$. Entonces V es un espacio hiperbólico y dim(V) es par.
\end{theorem}

\begin{proof}
Como $\psi$ es no degenerada entonces podemos asumir que $\psi \neq 0$.
Existen entonces, $\tilde{e}_1$ y $f_1$ no nulos, tales que $\psi(\tilde{e}_1,f_1)=c\neq0$.

Definimos, $e_1:=\frac{\tilde{e}_1}{c}$ entonces, $\psi(e_1,f_1)=1$.

Notar que $e_1$ y $f_1$, son linealmente independientes, pues si $e_1=c f_1$ con $c\neq 0$,
entonces $\psi(e_1,f_1)=\psi(c f_1,f_1)=0$


Sea $P_1=<e_1,f_1>$ el $\mathbb{Z}$-módulo generado por los elementos $e_1,f_1$
    
Dado que $\psi(e_1,f_1)=1$ y $\psi$ es alternante, tenemos que:
\begin{itemize}
 \item $\psi(e_1,e_1)=0$
 \item $\psi(f_1,f_1)=0$
 \item $\psi(f_1,e_1)=-1$
\end{itemize}

Por lo que, 
$$\psi_{|P_1} = 
\begin{bmatrix}
 1 & 0 &\\ 
 0& -1&
\end{bmatrix}
$$

Se ve así que $\psi_{|P_1}$ es no degenerada $\Rightarrow P_1$ es un plano hiperbólico.
\newline

Afirmación 2: Se tiene que  $V = P_1 \oplus P_1^{\bot}$

 donde $P_1^{\bot}= \{ \alpha \in V: \psi(\alpha,\beta)=0$  $\forall$  $\beta \in P_1 \} $ .En efecto,  
 sea v cualquier elemento en V.
 Definimos 
 $$v_1:=\psi(v,f_1)e_1-\psi(v,e_1)f_1$$
 $$v_2:=v-v_1$$
 Notar que $v_1 \in <e_1,f_1>=P_1$ y $v_2 \in P_1^{\bot}$, pues 
 
 $$\begin{aligned}
 \psi(v_2,e_1)=\psi(v,e_1)-\psi(v_1,e_1)=&\\
 \psi (v,e_1)-[\psi(v,f_1) \psi(e_1,e_1) + \psi(v,e_1) \psi(f_1,e_1)]=&\\
 \psi(v,e_1)- \psi(v,e_1)=&0.
 \end{aligned}$$
 Análogamente $\psi(v_2,f_1)=0$.
 
 
 Además, $P_1  \cap P_1^{\bot} = \{0\}$, pues si  $\tilde{v} \in  P_1 \cap P_1^{\bot}$, entonces
 $$\begin{aligned}
 \psi(\tilde{v},v_1)&=0 \;\forall v_1 \in P_1\\ 
 \psi(\tilde{v},v_2)&=0 \;\forall v_2 \in P_1^{\bot},
 \end{aligned}$$
 
 con lo cual, $\psi(\tilde{v},v)=0$ $\forall v \in V=P_1 + P_1^{\bot}$
 
 
Hemos visto que, $V=P_1 \oplus P_1^{\bot}$, con $P_1^{\bot}=\{v \in V : \psi(v,e_1)=0=\psi(v,f_1)\}$

Por inducción, aplicamos la hipótesis a $\psi|_{P_1^{\bot}}$ obteniendo los pares $(e_i,f_i)$ con $i \in \mathbb{N}$

Como $V$ tiene dimensión finita, existe un $n \in \mathbb{N}$ tal que  $\beta=\{e_1,f_1,...,e_n,f_n\}$ es una base de $V$.
En esta base, 
$$\psi = 
\begin{bmatrix}
 0 & 1 & & & & & & &\\ 
 -1& 0 & & & & & & &\\
 & & 0 & 1 & & & & &\\
 & & -1 & 0 & & & & &\\
 & &  &  & & \ddots & & &\\
 & &  &  & & & 0 & 1 &\\
 & &  &  & & & -1 & 0 &\\ 
\end{bmatrix}
$$

y se ve que la $dim(V)$ es par. 
\end{proof}
\newline


De ahora en adelante, sea $\psi :\mathbb{R}^{2n} \times \mathbb{R}^{2n}\mapsto \mathbb{R}$ dada por 
$$ \begin{aligned}
\psi((x,y),(x',y'))=y.x'-x.y' ,
\end{aligned}$$
donde $x,y,x',y' \in \mathbb{R}^n$ con  $x.y$ el producto escalar usual de $\mathbb{R}^n$.

\subsection{Grupo de Heisenberg}
\begin{definition}
 El grupo de Heisenberg $\mathbb{H}_n$, es el grupo de Lie cuya variedad subyacente es $\mathbb{R}^{2n} \times \mathbb{R}$ y 
 cuya operación de grupo está definida por: 
 $$(x,y,t)(x',y',s)=(x+x'+y+y',t+s + \frac{1}{2} \psi((x,y),(x',y'))),$$
 donde $x,x',y,y' \in \mathbb{R}^{n}$ , s,t $\in \mathbb{R}$.
\end{definition}



Observación:
El grupo de Heisenberg también puede ser visto como  $\mathbb{C}^{n}\times R,$ con
ley de grupo dada por 
$$\begin{aligned}
(v,t)(v',s)=(v+v',t+s+ \frac{1}{2} \psi(v,v')),
\end{aligned}$$
donde $v=(x,y),v'=(x',y') \in \mathbb{C}^n$ y $\psi (v,v')=Im(v.\overline{v'}).$

El centro de $H_{n}$ viene dado por Z=(0,t), t $\in \mathbb{R}.$

\begin{definition}
 Recordemos que, dados dos elementos $g,h$ en un grupo $G$, el conmutador de $g$ y $h$ es ${[}g,h{]}:=g h g^{-1} h^{-1}$.
 
 Llamaremos subgrupo conmutador $[G,G]$ al grupo generado por todos los conmutadores.
\end{definition}
Notemos que, dados $g=(v,t),h=(v',s) \in \mathbb{H}_n$
$$
{[}g,h{]}=g h g^{-1} h^{-1}=(v+v',t+s+\frac{1}{2}\psi(v,v')) (-v-v',-t-s + \frac{1}{2}\psi(v,v'))=(0,\psi(v,v'))
$$

\section{Automorfismos del grupo de Heisenberg}
\begin{definition}
 Recordemos que un automorfismo $\Phi: \mathbb{H}_n \mapsto \mathbb{H}_n$ es un isomorfismo lineal tal que 
 $$\begin{aligned}
    f({[}(v,t),(v',s){]})={[}f(v,t),f(v',s){]}
 \end{aligned}$$
 

\end{definition}

Sea $A \in GL(2n,\mathbb{R})$ tal que  $\psi(v,v')=\psi(Av, Av')$, y  $f:\mathbb{H}_n \mapsto \mathbb{H}_n$ la aplicación
que tiene como matriz asociada 
$$[f] = 
\begin{bmatrix}
A & 0\\
0 & 1
\end{bmatrix},$$

entonces f es un automorfismo, pues,



Dado que $A \in GL(2n,\mathbb{R})$, f es biyectiva.

Y finalmente,
$$\begin{aligned}
f({[}(v,t),(v',s){]})=f(0,\psi(v,v'))&=\\
(0,\psi(v,v'))=(0,\psi(Av,Av'))&=\\
{[}(Av,t),(Av',s){]}&={[}f(v,t),f(v',s){]}.\\
\end{aligned}$$

Esta última observación motiva la siguiente definición 
\begin{definition}
 Definimos el grupo simpléctico $Sp(2n,\mathbb{R})$ como ,
 $$
 Sp(2n,\mathbb{R})=\{ A \in GL(2n,\mathbb{R}): \psi(v,v')=\psi(Av,Av') \forall v,v' \in \mathbb{C}^n\}.
 $$
\end{definition}
Observemos que este es el subgrupo de automorfismos que actuan trivialmente en el centro.


Veremos más en detalle, el grupo de automorfismos de $\mathbb{H}_n$, $\mathbb{A}_n$.
Todo $\alpha \in \mathbb{A}_n$ se puede ver como una transformación lineal en $\mathbb{R}^{2n}$,

$$\alpha=
\begin{bmatrix}
A & c\\
b & a
\end{bmatrix},$$
con $A \in  \mathbb{R}^{2n} \times \mathbb{R}^{2n}$, $a \in \mathbb{R}$, $b,c \in \mathbb{R}^{2n}$.

El hecho de que $\alpha$ es un autormofismo, nos brinda más información sobre estos elementos.

Sabemos que, $\alpha(h) \in Z \forall h \in Z$ , donde $Z$ denota el centro de $\mathbb{H}_n$.
Esto nos indica que $c=0$.
Además, dado que $\alpha$ es inyectiva, $a\neq0$.

Sobre $A$, tenemos la condición $a \psi(v,v')=\psi(A v,A v')$, pues
$$\begin{aligned}
(0,\psi(A v,A v'))={[}\alpha(v,t),\alpha(v',s){]}=\alpha{[}(v,t),(v',s){]}=(0,a \psi(v,v')).
\end{aligned}$$
Además, dado que $\alpha$ es invertible, $A \in GL(2n,\mathbb{R})$.

Ahora veamos el caso en que $a >0$.

Si definimos $\delta_a (v,t) :=(a v,a^2 t)$, tenemos que 
$$\alpha=
\begin{bmatrix}
A & 0\\
b & a
\end{bmatrix}=
\delta_{a^{1/2}}
\begin{bmatrix}
S & 0\\
0 & 1
\end{bmatrix}
\begin{bmatrix}
I & 0\\
\tilde{b} & 1
\end{bmatrix},
$$
con $\tilde{b}=\frac{b}{a}$ y  $S$ tal que $S(v)=a^{-1/2} A(v)$.
De esta última condición sobre $S$, se sigue que $S \in Sp(2n,\mathbb{R}).$

Para el caso, donde $a<0$, introducimos $\theta(v,t):=(\overline{v},-t)$.

Dado que $\theta=\theta^{-1}$,
$$\alpha=
\begin{bmatrix}
A & 0\\
b & a
\end{bmatrix}=
\theta 
\theta
\begin{bmatrix}
A & 0\\
b & a
\end{bmatrix}=
\theta
\begin{bmatrix}
\tilde{A} & 0\\
b & -a
\end{bmatrix}.
$$
Dado que $\alpha$ y $\theta$ son automorfismos, $\theta \alpha$ lo es. Además $-a>0$, por lo que tenemos que $\theta \alpha$ es 
un automorfismo como el visto en el caso anterior, y por ende,
$$\alpha=
\theta
\delta_{a^{1/2}}
\begin{bmatrix}
S & 0\\
0 & 1
\end{bmatrix}
\begin{bmatrix}
I & 0\\
\tilde{b} & 1
\end{bmatrix}.
$$

\section{Descripción de lattices en el grupo de Heisenberg}

\begin{definition}
 Recordemos que una lattice de $\mathbb{C}^n$ es un subgrupo discreto D de $\mathbb{C}^n$ tal que $\mathbb{C}^n$/D es compacto.
\end{definition}

\begin{definition}
 Llamaremos una lattice de $\mathbb{H}_n$, a un subgrupo discreto $\Gamma$ de $\mathbb{H}_n$ tal que $\mathbb{H}_n/\Gamma$ es compacto.
\end{definition}

Notemos, que  $\Gamma \cap Z$ es un subgrupo,discreto, no trivial, de $Z$ y 
por ende existe un único real positivo $\beta(\Gamma)$ tal que:
$$
\Gamma \cap Z=\beta(\Gamma) \mathbb{Z}.
$$

Consideremos $\pi :H_{n}\rightarrow \mathbb{C}^{n}$  definida por $\pi (v,t)=v$,
entonces por \cite{Ma}, $\pi(\Gamma)$ es una lattice de $\mathbb{C}^n$.

Además, el hecho de que
$$\begin{aligned}
{[} \Gamma,\Gamma {]} \subseteq \Gamma \cap Z = \beta(\Gamma) \mathbb{Z}
\end{aligned}$$
impone algunas condiciones especiales a $\pi(\Gamma)$.
Sean $h=(v,t),g=(v',s) \in \Gamma$, entonces ${[}h,g{]}=h^{-1}g^{-1}hg=(0,\psi(v,v'))$ 
y por ende $\psi(v,v') \in  \beta(\Gamma) \mathbb{Z}$.
Es más, despues veremos que $\psi(\pi(\Gamma),\pi(\Gamma))=\beta(\Gamma) \mathbb{Z}$.
Esto motiva la siguiente definición:

\begin{definition}
 Llamaremos a una lattice D de $\mathbb{C}^n$, lattice de Heisenberg, cuando $\psi(D,D)=l \mathbb{Z} $ para algún $l > 0$.
 Denotaremos a los lattice de Heisenberg de $\mathbb{C}^n$ como el conjunto $HL(\mathbb{C}^n)$
\end{definition}



\begin{lemma} \label{1.7}
 Si una lattice D de $\mathbb{C}^n$  satisface que $\psi(D,D) \subseteq  l \mathbb{Z}$, $l>0$, entonces 
 existe un entero positivo $l_1$ tal que $\psi (D,D)=l l_1 \mathbb{Z}$, es decir D $\in HL(\mathbb{C}^{n}).$
\end{lemma}
\begin{proof}
 Sin pérdida de generalidad podemos asumir que l=1, pues si $l\neq1$, definimos  
 $f:D\mapsto \tilde{D}$ por $f(d):=d\frac{1}{\sqrt{l}}$
y luego $\psi (\tilde{D},\tilde{D})\subseteq \mathbb{Z}$
 
Sea $l_1$ el menor entero positivo para el cual existen $
v_1,v_1' \in D$ tal que, $l_1=\psi (v_1,v_1')$ con 
$v_1,v_1' \in D$.
 Definimos $<v_1,v_1'>^\bot=\{v \in D: \psi(v,v_1)=\psi(v_1',v)=0\}$
 \newline
 
 Afirmación 1: $D=<v_1,v_1'>\oplus <v_1,v_1'>^\bot$ . En efecto, sea $v \in D$,
 
 por el algoritmo Euclideo, $\psi(v_1,v)=m l_1 +r$ con $m,r \in \mathbb{Z}$ y $0 \leq r < l_1 \Rightarrow $
 
 $r=\psi(v_1,v)-m l_1=\psi(v_1,v)-m \psi(v_1,v_1')=\psi(v_1,v-mv_1')$
 
 
Luego $r=0$, por la elección de $l_1$, esto es $\psi(v_1,v-mv_1')=0$
 
 De esta forma $v-mv_1' \in <v_1>^\bot$.
 Analogamente, $v-m'v_1 \in <v_1'>^\bot$  con $m'\in \mathbb{Z}$.
 
 Notar que,
 $$\begin{aligned}
 \psi(v-(m v_1' + m' v_1),v_1)=\psi(v-m v_1',v_1) + \psi(-m' v_1,v_1)=0=\\
 \psi(v_1',v-(m v_1'+m'v_1))=\psi(v_1',v-m'v_1)+ \psi(v_1',-m v_1') \therefore \\
 v-(m v_1'+m'v_1) \in <v_1,v_1'>^\bot
 \end{aligned}$$
 
 Finalmente, para cualquier $v\in D, v=v-(m v_1'+ m'v_1)+(m v_1' + m'v_1)$ 
 \newline
  
 
 Afirmación 2: $l_1 | \psi(v,v')$ $\forall v,v' \in <v_1,v_1'>^\bot$.
 
 Pues, $\psi(v,v')=m l_1 + r$ con $m,r \in \mathbb{Z}$ y  $0\leq r < l_1 \Rightarrow $
 
 $r= \psi(v,v') - m \psi(v_1,v_1')=\psi(v,v')-m\psi(v_1,v_1')+ \psi (v_1,v')+ \psi(v,-m v_1') 
 = \psi(v+v_1,v'-mv_1') \Rightarrow $ r=0 (por la elección de $l_1$) 
 
 Finalmente afirmo que $\psi(D,D)=l_1 \mathbb{Z}$ 
 
 $\supseteq$)Esta inclusión es inmediata: pues si $m \in \mathbb{Z} \Rightarrow \psi(v_1,m v_1') = m l_1$
 
 $\subseteq$) Sean $w,w' \in D$ ,por afirmación 1 ,$w = w_1 +w_2$  y $w'=w_1' +w_2'$
 
  con $w_1,w_1' \in <v_1,v_1'>$ y  $w_2,w_2' \in <v_1,v_1'>^\bot$ 
 $$\begin{aligned}
 \psi(w,w')&=\\
 \psi(w_1,w_1')+ \psi(w_1,w_2')+ \psi(w_2,w_1')+\psi(w_2,w_2')&=\\
 \psi(w_1,w_1')+\psi(w_2,w_2')& \stackrel{*}{=} \\ 
 \psi(n v_1+n' v_1',m v_1+m' v_1') + k l_1 &=\\
 n m' \psi(v_1,v_1')+n' m \psi(v_1',v_1)+ k l_1 &= (n m' - n'm+k)l_1\\
 \end{aligned}$$
 \newline 
 
 (*) $w_1,w_2 \in <v_1,v_1'>$ y afirmación 2
 
 Notar que el lattice original $\psi(D,D)= l l_1 \mathbb{Z} $
\end{proof}

\begin{lemma} \label{1.8}
 Un $D \in HL(\mathbb{C}^n)$ se puede escribir como una suma directa $\psi-ortogonal : D=<v_1,v_1'>\oplus<v_2,v_2'>...<v_n,v_n'>$
 de $\mathbb{Z}$-módulos bidimensionales $<v_j,v_j'>$. Sea $l_j=\psi(v_j,v_j')$; entonces la descomposición puede ser tomada 
 de forma tal que $l_j \in \mathbb{R}^{>0}$ y $\frac{l_{j+1}}{l_j}  \in \mathbb{Z}$.
 
 Es más, estos $l_j$ están unívocamente determinados por D.
\end{lemma}

\begin{proof}
Por el lema \ref{1.7} sabemos que existen $l_1$ entero no negativo y $l \in \mathbb{R}$ tales que  $\psi(D,D)  = l l_1 \mathbb{Z}$. Sin pérdida de generalidad podemos asumir l=1.
Luego existe $v_1,v_1' \in D$ tal que  $l_1=\psi(v_1,v_1')$.

Sea $P_1:=<v_1,v_1'>$ el $\mathbb{Z}$-módulo bidimensional generado por $v_1,v_1'$.

Notar que $$
\psi|_{<v_1,v_1'>} = \begin{bmatrix}
                              0 & l_1 &\\
                              -l_1 & 0 &
                             \end{bmatrix}
$$

En la prueba del lema \ref{1.7} vimos que $D=<v_1,v_1'> \oplus <v_1,v_1'>^\bot$.

Como $\psi(P_1^\bot,P_1^\bot) \subseteq l_1 \mathbb{Z}$ podemos usar recursivamente el mismo razonamiento anterior 
para obtener los pares $(v_j,v_j')$, $j=2...m$ tales que 
$D=<v_1,v_1'>\oplus<v_2,v_2'>...<v_m,v_m'>$

Si $m<n$ $\Rightarrow \mathbb{R}^{2n}/D=\mathbb{R}^{2n}/\oplus_{i=1}^m P_i \backsimeq \oplus_{i=1}^m (\mathbb{R}^{2n}/P_i) \oplus \mathbb{R}^{2(n-m)} $\newline
Luego $\mathbb{R}^{2n}/ D$ no sería compacto , absurdo! pues D es un lattice.
\newline


Veamos que $l_j|l_{j+1}$.

Ya vimos que $D= P_1\oplus P_2 \oplus...\oplus P_j \oplus P_j^\bot$.

$l_{j+1}=\psi(v_{j+1},v_{j+1}')=m \psi(v_j,v_j') +r \Rightarrow$ 

$$\begin{aligned}
r&=\psi(v_{j+1},v_{j+1}')-m \psi(v_j,v_j') + 0 + 0\\
&=\psi(v_{j+1},v_{j+1}')-m \psi(v_j,v_j') +  \psi(v_{j+1},-mv_j') + \psi(v_j,v_{j+1}')\\
&=\psi(v_{j+1} + v_j, v_{j+1}'-m v_j') \\
\end{aligned}$$ 
con ($v_{j+1} + v_j) $, $(v_{j+1}'-m v_j') \in P_j \oplus P_j^\bot \Rightarrow
r=0$ pues $l_j$ es el más pequeño de $\psi(P_j \oplus P_j^\bot, P_j \oplus P_j^\bot)$
\newline

Queda por ver la unicidad.

Primero veamos que $l_1$ es único. Supongamos que  existe un $\tilde{l}_1$ tal que: 

$\psi(D,D)=\tilde{l}_1 \mathbb{Z}$ $\Rightarrow$ $\tilde{l}_1 | l_1$ y $l_1 | \tilde{l}_1$ $\therefore$ $\tilde{l}_1=l_1$

La prueba continua por inducción sobre la dimensión de D.

Para Dim(D)=2 la prueba resulta de la unicidad de $l_1$

Dado n $\in \mathbb{N}$ , supongamos que los $l_i$ son únicos para los lattice de Heisenberg de dimensión 2n-2.
Para D con dimensión 2n, defino 
$$\tilde{D}:=D/(v_1 \mathbb{Z} \oplus v_1' \mathbb{Z})$$
$\tilde{D}$ es un lattice de Heisenberg, de dimensión 2n-2, por lo que sus $2\leq l_i\leq n$ asociados son únicos
(por hipótesis inductiva). Además el $l_1$ es único por lo visto en el primer párrafo de la unicidad.

 \end{proof}
 
 \begin{corollary}
  $\pi L = HL(\mathbb{C}^n)$
 \end{corollary}
 \begin{proof}
$\subseteq)$ Sea $\Gamma \in L$

$[\Gamma,\Gamma] \subseteq \Gamma \cap Z \Rightarrow$ es un subgrupo discreto de $Z (\simeq \mathbb{R})$

Además, $(0,\psi(\pi(\Gamma),\pi(\Gamma))) = \{(0,\psi(\pi(h),\pi(h'))): h,h' \in \Gamma\} = [\Gamma,\Gamma] \subseteq (0,\beta(\Gamma) \mathbb{Z})$

Entonces tenemos que:
\begin{itemize}
 \item $\psi(\pi(\Gamma),\pi(\Gamma)) \subseteq \beta(\Gamma) \mathbb{Z}$
 \item $\psi(\pi(\Gamma),\pi(\Gamma))$ es un lattice de $\mathbb{C}^n (por \cite{Ma}$)
\end{itemize}

Con lo cual, estamos en condiciones de aplicar el lema\ref{1.7}, y concluir que 

$\psi(\pi(\Gamma),\pi(\Gamma))=l_1 \beta(\Gamma) \mathbb{Z} \therefore \pi(\Gamma) \in HL(\mathbb{C}^n) $
\newline

$\supseteq)$ $D\in HL(\mathbb{C}^n) \Rightarrow D=<v_1,v_1'>\oplus ... \oplus <v_n,v_n'>$ por (\ref{1.8}) 

Defino $\Gamma$  como el grupo generado por
$${(v_1,0),(v_2,0),...,(v_n,0),(v_1',0),(v_2',0),...,(v_n',0)}.$$

Más explicitamente, $\Gamma=\{(\sum_{i=1}^n a_i v_i + b_i v_i', \frac{1}{2} m l_1):a_i,b_i,m \in \mathbb{Z} \}$

Es inmediato ver que $\Gamma$ es un subgrupo discreto y que $\mathbb{H}_n / \Gamma$ es compacto.

 \end{proof}

 \begin{corollary}
  Sea $\Gamma  \in L$ entonces, $\psi(\pi(\Gamma),\pi(\Gamma))=l_1 \beta(\Gamma) \mathbb{Z}$  
 \end{corollary}
 \begin{proof}
  Por el corolario anterior, $D=\pi(\Gamma) \in HL(\mathbb{C}^n) \therefore \psi(D,D)=l_1 \beta \mathbb{Z}$  
 \end{proof}


\begin{example} 
 Definimos,
 $$Z_n^{\#}:= \{ \ell=(l_1,l_2,...,l_n) \in \mathbb{N}^n:l_j|l_{j+1} \}$$
 
 Sea $e_j$, $1\leq j \leq 2n $,  la base canónica de $\mathbb{R}^{2n}$.Para $\ell$, definimos:
 
 $$D(\ell):=<e_1,....e_n,l_1 e_{n+1},l_2 e_{n+2}...,l_n e_{2n}>$$
  
 
 Se ve facilmente que $\psi(e_j,e_{j+n})=1$ para $j=1...n$
\end{example}

 Observación: En el lema $\ref{1.8}$ vimos que si $D(\ell) \in HL(\mathbb{C}^n)$ entonces admitía una descomposición de la forma 
 $<v_1,v_1'>\oplus ... <v_n,v_n'>= D(\ell)$
 
 En el ejemplo anterior, $v_i=e_i$ y $v_i'=l_i e_{i+1}$
 

\begin{theorem}
 Para cada $D \in HL(\mathbb{C}^n)$ existe un único $\ell \in Z_n^{\#}$, un único $l \in \mathbb{R}_{>0}$ y un $\alpha \in SP(2n,\mathbb{R})$ tal que:
 
 $D=\alpha(l^{1/2} D(\ell))$
\end{theorem}

\begin{proof}
 Si $D \in HL(\mathbb{C}^n)$,entonces $ \psi(D,D)=l l_1 \mathbb{Z}$, con $l \in \mathbb{R}_{>0}$ y $l_1 \in \mathbb{N}$ 
 
 Por el lema $\ref{1.8}$, $\tilde{D}:= \frac{1}{l^{1/2}} D =  <v_1,v_1'> \oplus...\oplus <v_n,v_n'>$
 con los números naturales $l_i=\psi(v_i,v_i')$ univocamente determinados.

 Definimos $\ell:=(l_1,...,l_n) \in \mathbb{Z}_n^{\#}$
 
 %Sea $v:=(v_1,...,v_n)$ y $v'=(v_1',...,v_n')$
  
 
 Sea $\alpha$ la transformacion lineal sobre $\mathbb{R}^{2n}$ definida por, 
 $$
 \begin{aligned}
 \alpha(e_j)=v_j,\\
 \alpha(e_{j+n})=\frac{v_{j+1}}{l_j},\\
 \end{aligned}$$
 
 con j=1..n.
 
 
Es fácil ver que $\psi(\alpha e_k,\alpha e_{k+n})=\psi(e_k,e_{k+n})$, luego $\alpha \in  SP(2n,\mathbb{R})$.

Con lo que $\tilde{D} = \alpha(D(\ell)) $

Pero entonces, $\tilde{D}=\frac{1}{l^{1/2}} D = \alpha(D(\ell)) \Rightarrow D=\alpha(l^{1/2} D(\ell))$
\end{proof}

\begin{definition}
 Llamaremos $\Gamma(\ell)$ al grupo generado por,
 $${(e_1,0),(e_2,0),...,(e_n,0),(l_1 e_{n+1},0),(l_2 e_{n+2},0),...,(l_n e_{2n},0)},$$
 con $\ell=(l_1,...l_n) \in \mathbb{Z}_n^{\#}$
\end{definition}

\begin{corollary}
 Sea $\Gamma \in L(\mathbb{H}_n)$, entonces existe un único $\ell \in  \mathbb{Z}_n^{\#}$, un único $l> 0$  y $\alpha \in Sp(2n,\mathbb{R})$ 
 tal que:
 
 $\Gamma = \delta_{l^{\frac{1}{2}}} \alpha (\Gamma(\ell))$
\end{corollary}
\begin{proof}
 Recordemos que ${[}\Gamma,\Gamma{]}=l l_1 \mathbb{Z}$.
 
 Sea $D:=\pi(\Gamma)=\alpha(l^{1/2} D(\ell))$.
 Proponemos, 
 $$
 \tilde{\Gamma}=   
  \delta_{l^{1/2}}
  \begin{bmatrix}  
    \alpha & 0\\ 
    0& 1
  \end{bmatrix}
  \Gamma(\ell).
$$
 Con $\tilde{\Gamma}$ asi definida, tenemos que,
 
 $$  \pi(\tilde{\Gamma})= 
 \pi(
 \delta_{l^{1/2}}
 \begin{bmatrix}
    \alpha & 0\\ 
    0& 1
  \end{bmatrix}
  \Gamma(\ell)
  )=
  l^{1/2}\alpha(\pi(\Gamma(\ell)))=
  l^{1/2}\alpha(D(\ell))=D=\pi(\Gamma)
  $$
  
  $${[}\tilde{\Gamma},\tilde{\Gamma}{]}=l l_1 \mathbb{Z}, $$
 
 por lo que concluimos que $\tilde{\Gamma}=\Gamma$.
\end{proof}





 \clearpage
  \begin{thebibliography}{1}
  \bibitem[To]{To} Tolimieri R.   \emph{``Heisenberg manifolds and Theta functions''}, Trans. of the A.M.S Vol 239, 1978. (293-319)
  \bibitem[Th]{Th} Thangavelu S.  \emph{``Harmonic analisys on Heisenberg nilmanifolds''}, Real analisys and its applications
  \bibitem[Ma]{Ma} Malcev, A. T.  \emph{``On a class of homogeneous spaces''}, Izv. Akad. Nauk SSSR Ser. Mat. 13(1949)
  \bibitem[La]{La} Lang S.  \emph{``Algebra''},3. ed., Springer, 2004
  \bibitem[Hu]{Hu} HungerfordT. W.  \emph{``Algebra''},V. 73,Springer, 2003
  \bibitem[Ho]{Ho} Hoffman K.,Kunze R.  \emph{``Linear Algebra''},Prentice-Hall, 1973  
  \end{thebibliography}
  
  
  
\end{document}


