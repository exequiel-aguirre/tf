\documentclass[12pt]{article}
\usepackage{lineno}
\usepackage[spanish]{babel}
\usepackage{amsmath}
\usepackage{amsfonts}
\usepackage[utf8]{inputenc}
\usepackage{pslatex}
\usepackage{theorem}
\newtheorem{theorem}{Teorema}
\newtheorem{lemma}{Lema}
\newtheorem{proposition}{Proposición}
\newtheorem{note}{Nota}
\newtheorem{definition}{Definición}

\author{Exequiel Aguirre}
\title{Subgrupos discretos del grupo de Heisenberg }
\date{}
% document start
\begin{document}
%\linenumbers
\maketitle

\begin{abstract}
La teoría de series de Fourier es la versión clásica del análisis armónico en el toro,
$\mathbb{R}^n /\mathbb{Z}^n$, donde $\mathbb{Z}^n$ denota el subgrupo discreto de $\mathbb{R}^n$ de
las n-uplas de números enteros.\newline
Si en lugar de $\mathbb{R}^n$ , consideramos el grupo de Heisenberg 2n+1 dimensional $H_n$, un tema
de interés actual es desarrollar el análisis armónico asociado a $H_n/\Gamma$, donde $\Gamma$ 
denota el subgrupo discreto de $H_n$. Ver \cite{Th} \newline
El objetivo primero de este trabajo es estudiar la descripción hecha por R. Tolimieri \cite{To}.
Posteriormente dar ejemplos de subgrupos discretos en el grupo de tipo Heisenberg, cuyo centro tiene
dimensión 2.

\end{abstract}


 

\section{Preliminares}
\begin{definition} 
 Un plano hiperbólico para una forma alternante es un espacio vectorial de dimensión 2 no degenerado,
  i.e. $\psi|_P$ sea no degenerada. 
\end{definition}

\begin{definition}   
 Decimos que un espacio vectorial V es hiperbólico si se puede escribir como suma ortogonal (directa ) de
 planos hiperbólicos
\end{definition}

\begin{theorem}
 Sea $\psi$ una forma bilineal, alternante, no degenerada, $\psi:VxV\mapsto \mathbb{R}$ con V, un espacio 
 vectorial de dimensión finita.
 Entonces: V es un espacio hiperbólico y dim(V) es par.
\end{theorem}


Demostración:
Como $\psi$ es no degenerada entonces puedo asumir que $\psi \neq 0$.
Existen entonces, $\tilde{e}_1$ y $f_1$ no nulos, tales que $\psi(\tilde{e}_1,f_1)=c\neq0$.
Defino, $e_1=\frac{\tilde{e}_1}{c}$ entonces, $\psi(e_1,f_1)=1$

Afirmación 1: $e_1,f_1$, son linealmente independientes.
Supongamos que  $c_1 e_1 +c_2 f_1 = 0$ con $c_1,c_2 \in \mathbb{R}$  donde $c_1\neq0$ o  $c_2\neq0$
Caso $c_1\neq0$:
Entonces tenemos que,
$0=\psi(e_1,e_1)=\psi(e_1,- \frac{c_2}{c_1} f_1)= - \frac{c_2}{c_1} \psi(e_1,f_1)=- \frac{c_2}{c_1} \Rightarrow c_2=0$
$\Rightarrow c_1 e_1 = 0$ absurdo!
Análogamente se puede comprobar que el caso $c_2\neq0$ también conduce a un absurdo, por lo que queda probada la
afimación.

Entonces tiene sentido definir $P_1=<e_1,f_1>$
    
Dado que $\psi(e_1,f_1)=1$ y $\psi$ es alternante, tenemos que:
\begin{itemize}
 \item $\psi(e_1,e_1)=0$
 \item $\psi(f_1,f_1)=0$
 \item $\psi(f_1,e_1)=0$
\end{itemize}

Por lo que, 
$$\psi|_{P_1} = 
\begin{bmatrix}
 1 & 0 &\\ 
 0& -1&
\end{bmatrix}
$$

Se ve así que $\psi|_{P_1}$ es no degenerada $\Rightarrow P_1$ es un plano hiperbólico.

Por Lema 1: $V=P_1 \oplus P_1^{\bot}$, con $P_1^{\bot}={v \in V : \psi(v,e_1)=0=\psi(v,f_1)}$
Por inducción, aplico la hipotesis a $\psi|P_1^{\bot}$ y obtengo los pares $(e_i,f_i)$ con $i \in \mathbb{N}$

Entonces sea $\beta$ la base: $\beta={e_1,f_1,...,e_n,f_n}$.
En esta base, 
$$\psi = 
\begin{bmatrix}
 0 & 1 & & & & & & &\\ 
 -1& 0 & & & & & & &\\
 & & 0 & 1 & & & & &\\
 & & -1 & 0 & & & & &\\
 & &  &  & & \ddots & & &\\
 & &  &  & & & 0 & 1 &\\
 & &  &  & & & -1 & 0 &\\ 
\end{bmatrix}
$$

De $\beta$ se ve que la dim(V) es par.


 \section{Referencias}
  \begin{thebibliography}{1}
  \bibitem[To]{To} Tolimieri, R.   \emph{"Heisenberg manifolds and Theta functions"}, Trans. of the A.M.S Vol 239, 1978. (293-319)
  \bibitem[Th]{Th} Thangavelu, S.  \emph{"Harmonic analisys on Heisenberg nilmanifolds"}, Real analisys and its applications
  \end{thebibliography}


\end{document}


