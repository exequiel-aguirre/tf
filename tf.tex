\documentclass[12pt]{article}
\usepackage{lineno}
\usepackage[spanish]{babel}
\usepackage{amsmath}
\usepackage{amsfonts}
\usepackage{amssymb}
\usepackage{color}
\usepackage[utf8]{inputenc}
\usepackage{pslatex}
\usepackage{theorem}
\newtheorem{theorem}{Teorema}
\newtheorem{lemma}{Lema}
\newtheorem{proposition}{Proposición}
\newtheorem{note}{Nota}
\newtheorem{definition}{Definición}
\newenvironment{proof}{\paragraph{Demostración:}}{\hfill$\square$}

\author{Exequiel Aguirre}
\title{Subgrupos discretos del grupo de Heisenberg }
\date{}
% document start
\begin{document}
%\linenumbers
\maketitle

\begin{abstract}
La teoría de series de Fourier es la versión clásica del análisis armónico en el toro,
$\mathbb{R}^n /\mathbb{Z}^n$, donde $\mathbb{Z}^n$ denota el subgrupo discreto de $\mathbb{R}^n$ de
las n-uplas de números enteros.\newline
Si en lugar de $\mathbb{R}^n$ , consideramos el grupo de Heisenberg 2n+1 dimensional $H_n$, un tema
de interés actual es desarrollar el análisis armónico asociado a $H_n/\Gamma$, donde $\Gamma$ 
denota el subgrupo discreto de $H_n$. Ver \cite{Th} \newline
El objetivo primero de este trabajo es estudiar la descripción hecha por R. Tolimieri \cite{To}.
Posteriormente dar ejemplos de subgrupos discretos en el grupo de tipo Heisenberg, cuyo centro tiene
dimensión 2.

\end{abstract}


 

\section{Preliminares}
\begin{definition} 
 Un plano hiperbólico para una forma alternante es un espacio vectorial de dimensión 2 no degenerado,
  i.e. $\psi|_P$ sea no degenerada. 
\end{definition}

\begin{definition}   
 Decimos que un espacio vectorial V es hiperbólico si se puede escribir como suma ortogonal (directa ) de
 planos hiperbólicos
\end{definition}

\begin{theorem}
 Sea $\psi$ una forma bilineal, alternante, no degenerada, $\psi:VxV\mapsto \mathbb{R}$ con V, un espacio 
 vectorial de dimensión finita.
 Entonces: V es un espacio hiperbólico y dim(V) es par.
\end{theorem}

\begin{proof}
Como $\psi$ es no degenerada entonces puedo asumir que $\psi \neq 0$.
Existen entonces, $\tilde{e}_1$ y $f_1$ no nulos, tales que $\psi(\tilde{e}_1,f_1)=c\neq0$.\newline
Defino, $e_1=\frac{\tilde{e}_1}{c}$ entonces, $\psi(e_1,f_1)=1$

Afirmación 1: $e_1,f_1$, son linealmente independientes.\newline
Supongamos que  $c_1 e_1 +c_2 f_1 = 0$ con $c_1,c_2 \in \mathbb{R}$  donde $c_1\neq0$ o  $c_2\neq0$\newline
Caso $c_1\neq0$:
Entonces tenemos que,
$0=\psi(e_1,e_1)=\psi(e_1,- \frac{c_2}{c_1} f_1)= - \frac{c_2}{c_1} \psi(e_1,f_1)=- \frac{c_2}{c_1} \Rightarrow c_2=0$
$\Rightarrow c_1 e_1 = 0$ absurdo!
Análogamente se puede comprobar que el caso $c_2\neq0$ también conduce a un absurdo, por lo que queda probada la
afimación.

Entonces tiene sentido definir $P_1=<e_1,f_1>$
    
Dado que $\psi(e_1,f_1)=1$ y $\psi$ es alternante, tenemos que:
\begin{itemize}
 \item $\psi(e_1,e_1)=0$
 \item $\psi(f_1,f_1)=0$
 \item $\psi(f_1,e_1)=-1$
\end{itemize}

Por lo que, 
$$\psi|_{P_1} = 
\begin{bmatrix}
 1 & 0 &\\ 
 0& -1&
\end{bmatrix}
$$

Se ve así que $\psi|_{P_1}$ es no degenerada $\Rightarrow P_1$ es un plano hiperbólico.

Por Lema 1: $V=P_1 \oplus P_1^{\bot}$, con $P_1^{\bot}=\{v \in V : \psi(v,e_1)=0=\psi(v,f_1)\}$
Por inducción, aplico la hipotesis a $\psi|P_1^{\bot}$ y obtengo los pares $(e_i,f_i)$ con $i \in \mathbb{N}$

Entonces sea $\beta$ la base: $\beta={e_1,f_1,...,e_n,f_n}$.
En esta base, 
$$\psi = 
\begin{bmatrix}
 0 & 1 & & & & & & &\\ 
 -1& 0 & & & & & & &\\
 & & 0 & 1 & & & & &\\
 & & -1 & 0 & & & & &\\
 & &  &  & & \ddots & & &\\
 & &  &  & & & 0 & 1 &\\
 & &  &  & & & -1 & 0 &\\ 
\end{bmatrix}
$$

De $\beta$ se ve que la dim(V) es par. 
\end{proof}

\begin{lemma}
 Sea V un espacio vectorial de dimensión finita, sobre un subcuerpo F del cuerpo de los números complejos y
 $\psi$ una forma bilineal alternante no degenerada, en V.
 Entonces existe un subespacio $P_1$, tal que \newline
 $V = P_1 \oplus P_1^{\bot}$
 donde $P_1^{\bot}= \{ \alpha \in V: \psi(\alpha,\beta)=0$  $\forall$  $\beta \in P_1 \} $
 \end{lemma}
 
 \begin{proof}
 Sea v cualquier elemento en V.
 Defino 
 $$v_1:=\psi(v,f_1)e_1-\psi(v,e_1)f_1$$
 $$v_2:=v-v_1$$
 Notar que $v_1 \in <e_1,f_1>=P_1$ y $v_2 \in P_1^{\bot}$, pues 
 $\psi(v_2,e_1)=\psi(v,e_1)-\psi(v_1,e_1)=\psi (v,e_1)-[\psi(v,f_1) \psi(e_1,e_1) + \psi(v,e_1) \psi(f_1,e_1)]=
 \psi(v_1,e_1)- \psi(v,e_1)=0$.\newline
 Análogamente $\psi(v_2,f_1)=0$.
 
 Es decir, que $v=v-v_1+v_1=v_2 + v_1$ con $v_2 \in P_1^{\bot}$ y $v_1 \in P_1$.\newline
 Además, $P_1  \cap P_1^{\bot} = \{0\}$, pues si  $\tilde{v} \in  P_1 \cap P_1^{\bot}$ entonces
 $\tilde{v} \in P_1 \Rightarrow \tilde{v}=a e_1 + b f_1 \Rightarrow 
 \psi(\tilde{v},e_1)=\psi(a e_1 + b f_1,e_1)=-b$ y $\psi(\tilde{v},f_1)=\psi(a e_1 + b f_1,f_1)=a $
 
 Es decir que podemos escribir $\tilde{v}=\psi(\tilde{v},f_1) e_1 - \psi(\tilde{v},e_1) f_1=0$ pues 
 $\tilde{v} \in P_1^{\bot}.$
 Así se ve que $V = P_1 \oplus P_1^{\bot}$ 
\end{proof} 



\section{?}

\begin{definition}
 Llamaremos una lattice de $H_n$, a un subgrupo discreto $\Gamma$ de $H_n$ tal que:
 $H_n/\Gamma$ es compacto.
\end{definition}

\begin{definition}
 Un lattice de $\mathbb{C}^n$ es un subgrupo discreto de $\mathbb{C}^n$ donde D/$\mathbb{C}^n$ es compacto.
\end{definition}


\begin{definition}
 Llamaremos a un lattice D de $\mathbb{C}^n$, lattice de Heisenberg, cuando $\psi(D,D)=l \mathbb{Z} $ para algún $l > 0$.
 Denotaremos a los lattice de Heisenberg de $\mathbb{C}^n$ como el conjunto $HL(\mathbb{C}^n)$
\end{definition}



\begin{lemma} \label{1.7}
 Si un lattice D de $\mathbb{C}^n$  satisface que $\psi(D,D) \subseteq  l \mathbb{Z}$, $l>0$ entonces D $\in HL(\mathbb{C}^n)$
\end{lemma}
\begin{proof}
 Sin pérdida de generalidad podemos asumir que l=1, pues si $l\neq1$,tomo un isomorfismo 
 $f:D \mapsto \tilde{D}$ donde $f(d):=d \frac{1}{\sqrt{l}}$. Entonces,\newline
 $\psi(\tilde{D},\tilde{D})=\{ \psi(\tilde{d},\tilde{d}): \tilde{d} \in \tilde{D} \}=
 \{ \psi(\frac{d}{\sqrt{l}},\frac{d}{\sqrt{l}}): d \in D \}=\{ \frac{1}{l} \psi(d,d):d\in D \} \newline
 \subset \{ \frac{1}{l} l a : a \in \mathbb{Z} \}= 1 \mathbb{Z} $
 
 Sea $l_1$ el entero positivo más pequeño de $\psi(D,D)$, donde  $ l_1 = \psi(v_1,v_1')$ con $v_1, v_1' \in D$.
 Definimos $<v_1,v_1'>^\bot=\{v \in D: \psi(v,v_1)=\psi(v_1',v)=0\}$
 
 Afirmación 1: $D=<v_1,v_1'>\oplus <v_1,v_1'>^\bot$ .
 Pues por el algoritmo Euclideo $\psi(v_1,v)=m l_1 +r$ con $m,r \in \mathbb{Z}$ y $0 \leq r < l_1$ ($\mathbb{Z}$ es dominio Euclideo)
 $\Rightarrow r=\psi(v_1,v)-m l_1=\psi(v_1,v)-m \psi(v_1,v_1')=\psi(v_1,v-mv_1')$\newline
 
 Luego r=0, pues si $r\neq 0$ entonces $0<\psi(v_1,v-mv_1')<l_1$ absurdo!\newline
 De esta forma $v-mv_1' \in <v_1>^\bot$.
 Analogamente, $v-m'v_1 \in <v_1'>^\bot$  con $m'\in \mathbb{Z}$ tal que  $\psi(v,v_1')=m'l_1 + r$
 
 Notar que $\psi(v-(m v_1' + m' v_1),v_1)=\psi(v-m v_1',v_1) + \psi(-m' v_1,v_1)=0=\psi(v_1',v-(m v_1'+m'v_1))=\psi(v_1',v-m'v_1)+
 \psi(v_1',-m v_1') \therefore  v-(m v_1'+m'v_1) \in <v_1,v_1'>^\bot$
 Finalmente,$v\in D, v=v-(m v_1'+ m'v_1)+(m v_1' + m'v_1)$ 
 Fin de la afirmación 
 
 Afirmación 2: $l_1 | \psi(v,v') \forall v,v' \in <v_1,v_1'>^\bot$ pues, $\psi(v,v')=m l_1 + r$ con $m,r \in \mathbb{Z}$ y 
 $0\leq r < l_1 \Rightarrow r= \psi(v,v') - m \psi(v_1,v_1')=\psi(v,v')-m\psi(v_1,v_1')+ \psi(v,-m v_1') + \psi (v_1,v')
 = \psi(v+v_1,v'-mv_1') \Rightarrow $ r=0 (por la definición de $l_1$)
 Fin de la afirmación
 
 
 Finalmente afirmo que $\psi(D,D)=l_1 \mathbb{Z}$ \newline
 $\supseteq$) Sea $m l_1 \in l_1 \mathbb{Z} \Rightarrow \psi(v_1,m v_1') = m \psi(v_1,v_1')= m l_1$\newline
 $\subseteq$) Sean w,w' $\in$ D ,por afirmación 1 ,\newline
 $w = w_1 +w_2$  y $w'=w_1' +w_2'$ con $w_1,w_1' \in <v_1,v_1'>$ y  $w_2,w_2' \in <v_1,v_1'>^\bot$\newline
 $\psi(w,w')=\psi(w_1,w_1')+ \psi(w_1,w_2')+ \psi(w_2,w_1')+\psi(w_2,w_2')=\psi(w_1,w_1')+\psi(w_2,w_2')=$ \footnote{$w_1,w_2 \in <v_1,v_1'>$ y afirmación 2}
 $=\psi(n v_1+n' v_1',m v_1+m' v_1') + k l_1 = n m' \psi(v_1,v_1')+n' m \psi(v_1',v_1)+ k l_1 = (n m' - n'm+k)l_1$
 
 
 Notar que el lattice original $\psi(D,D)= l l_1 \mathbb{Z} $
\end{proof}

\begin{lemma} \label{1.8}
 Un $D \in HL(\mathbb{C}^n)$ se puede escribir como una suma directa $\psi-ortogonal : D=<v_1,v_1'>\oplus<v_2,v_2'>...<v_n,v_n'>$
 de $\mathbb{Z}$-módulos bidimensionales $<v_j,v_j'>$. Sea $l_j=\psi(v_j,v_j')$, entonces la descomposición puede ser tomada 
 de forma tal que: $l_j \in \mathbb{R}^{>0} y l_j | l_{j+1}$
 Es más, estos $l_j$ están unívocamente determinados.
\end{lemma}

\begin{proof}
$\psi(D,D)  = l \mathbb{Z}$. Sin pérdida de generalidad podemos asumir l=1
Sea $l_1$ el mínimo entero positivo de $\psi(D,D)$, donde $l_1=\psi(v_1,v_1')$ con $v_1,v_1' \in D$. \newline
$v_1,v_1'$ son linealmente independientes. (pues si $v_1'=c v_1 \Rightarrow \psi(v_1,c v_1)=0$ ) entonces tiene sentido
pensar en el $\mathbb{Z}$-módulo bidimensional $P_1:=<v_1,v_1'>$

Notar que $$
\psi|_{<v_1,v_1'>} = \begin{bmatrix}
                              l_1 & 0 &\\
                              0 & -l_1 &
                             \end{bmatrix}
$$

En la prueba del lema \ref{1.7} vimos que $D=<v_1,v_1'> \oplus <v_1,v_1'>^\bot$
Como $\psi(P_1^\bot,P_1^\bot) \subseteq \mathbb{Z}$ podemos aplicar el mismo razonamiento del principio a $P_1^\bot$ 
para obtener los pares $(v_j,v_j')$

Como $D \in HL(\mathbb{C}^n)$ con $\mathbb{C}^n$ de dimensión finita,  obtengo una cantidad finita de pares,entonces:
$D=<v_1,v_1'>\oplus<v_2,v_2'>...<v_n,v_n'>$

Afirmación: m=n\newline
Sabemos que $m\leq n$, veamos que $m<n$ nos conduce a un absurdo\newline
Si $m<n$ $\Rightarrow \mathbb{R}^{2n}/D=\mathbb{R}^{2n}/\oplus_{i=1}^m P_i \backsimeq \oplus_{i=1}^m (\mathbb{R}^{2n}/P_i) \oplus \mathbb{R}^{2(n-m)} $\newline
Dado que $m<n \Rightarrow  mathbb{R}^{2(n-m)}$ no es compacto $\Rightarrow mathbb{R}^{2n}/D$ no es compacto, absurdo! pues D es un lattice
Fin afirmación

%Vimos en la demostración del lemma \ref{1.7}, que l_1 \ \psi(v,v') \forall v,v'  \in D
Veamos que $l_j|l_{j+1}$.\newline
Ya vimos que $D= P_1\oplus P_2 \oplus...\oplus P_j \oplus P_j^\bot$.\newline
$l_{j+1}=\psi(v_{j+1},v_{j+1}')=m \psi(v_j,v_j') +r \Rightarrow r=\psi(v_{j+1},v_{j+1}')-m \psi(v_j,v_j') + \psi(v_{j+1},v_{j+1}')
=\psi(v_{j+1},v_{j+1}')-m \psi(v_j,v_j') + \psi(v_{j+1},v_{j+1}') + \psi(v_{j+1},-mv_j') + \psi(v_j,v_{j+1}')
=\psi(v_{j+1} + v_j, v_{j+1}'-m v_j')$ con $v_{j+1} + v_j, v_{j+1}'-m v_j' \in P_j \oplus P_j^\bot \Rightarrow
r=0$ pues $l_j$ es el más pequeño de $\psi(P_j \oplus P_j^\bot, P_j \oplus P_j^\bot)$\newline

{\color{red} Queda por ver la unicidad}



 \end{proof}





 \section{Referencias}
  \begin{thebibliography}{1}
  \bibitem[To]{To} Tolimieri, R.   \emph{"Heisenberg manifolds and Theta functions"}, Trans. of the A.M.S Vol 239, 1978. (293-319)
  \bibitem[Th]{Th} Thangavelu, S.  \emph{"Harmonic analisys on Heisenberg nilmanifolds"}, Real analisys and its applications
  \end{thebibliography}


\end{document}


