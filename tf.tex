\documentclass[12pt]{article}
\usepackage{lineno}
\usepackage[spanish]{babel}
\usepackage{amsmath}
\usepackage{amsfonts}
\usepackage[utf8]{inputenc}
\usepackage{pslatex}
\usepackage{theorem}
\newtheorem{teorema}{Teorema}
\newtheorem{proposicion}{Proposición}
\newtheorem{nota}{Nota}

\author{Exequiel Aguirre}
\title{Subgrupos discretos del grupo de Heisenberg }
\date{}
% document start
\begin{document}
%\linenumbers
\maketitle

\begin{abstract}
La teoría de series de Fourier es la versión clásica del análisis armónico en el toro,
$\mathbb{R}^n /\mathbb{Z}^n$, donde $\mathbb{Z}^n$ denota el subgrupo discreto de $\mathbb{R}^n$ de
las n-uplas de números enteros.\newline
Si en lugar de $\mathbb{R}^n$ , consideramos el grupo de Heisenberg 2n+1 dimensional $H_n$, un tema
de interés actual es desarrollar el análisis armónico asociado a $H_n/\Gamma$, donde $\Gamma$ 
denota el subgrupo discreto de $H_n$. Ver \cite{Th} \newline
El objetivo primero de este trabajo es estudiar la descripción hecha por R. Tolimieri \cite{To}.
Posteriormente dar ejemplos de subgrupos discretos en el grupo de tipo Heisenberg, cuyo centro tiene
dimensión 2.

\end{abstract}


 

\section{Preliminares}


 \section{Referencias}
  \begin{thebibliography}{1}
  \bibitem[To]{To} Tolimieri, R.   \emph{"Heisenberg manifolds and Theta functions"}, Trans. of the A.M.S Vol 239, 1978. (293-319)
  \bibitem[Th]{Th} Thangavelu, S.  \emph{"Harmonic analisys on Heisenberg nilmanifolds"}, Real analisys and its applications
  \end{thebibliography}


\end{document}

