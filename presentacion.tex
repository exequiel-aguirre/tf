\documentclass{beamer}

\usepackage{amsmath}
\usepackage{graphicx}
\usepackage[utf8]{inputenc}
\usetheme{Boadilla}

\newtheorem{teorema}{Teorema}
%\newtheorem{lemma}{Lema}
%\newtheorem{proposition}{Proposición}
%\newtheorem{corollary}{Corolario}
%\newtheorem{example}{Ejemplo}
%\newtheorem{note}{Nota}
\newtheorem{definicion}{Definición}
%\newenvironment{proof}{\paragraph{Demostración:}}{\hfill$\square$}

\author{Alumno: Exequiel Aguirre \\ Directora: Dra. Linda V. Saal}
\title{Subgrupos discretos del grupo de Heisenberg }
\date{Julio 2014}

\begin{document}
\maketitle


\begin{frame}{Introducción}
\framesubtitle{Motivación}
La teoría de series de Fourier es la versión clásica del análisis armónico en el toro,
$\mathbb{R}^n /\mathbb{Z}^n$.

Si en lugar de $\mathbb{R}^n$ , consideramos el grupo de Heisenberg 2n+1 dimensional $\mathbb{H}_n$, un tema
de interés actual es desarrollar el análisis armónico asociado a $\mathbb{H}_n/\Gamma$, donde $\Gamma$ 
denota un subgrupo discreto de $\mathbb{H}_n$. 
\end{frame}


\begin{frame}{Preliminares}
Recordamos que una forma bilineal sobre un espacio vectorial real $V$ de dimensión finita,$\psi :V\times V\rightarrow R$
es antisimétrica si $\psi(v,w)=-\psi(w,v)$ $\forall v,w \in V$  y se dice no degenerada si 
$\psi(v,v')=0$ $\forall v' \in V \Rightarrow v = 0$  y 
$\psi(v,v')=0$ $\forall v \in V \Rightarrow v' = 0$
\end{frame}

\begin{frame}{Preliminares}
\begin{teorema} \label{Lang-8.1}
 Sea V, un espacio vectorial real de dimensión finita, $\psi :V\times V\rightarrow R$ una forma bilineal, antisimétrica, no degenerada.
  Entonces dim(V) es par y existe una base ordenada de V en la que la matriz de $\psi$,
 
$$\psi = 
\begin{bmatrix}
 0 & 1 & & & & & & &\\ 
 -1& 0 & & & & & & &\\
 & & 0 & 1 & & & & &\\
 & & -1 & 0 & & & & &\\
 & &  &  & & \ddots & & &\\
 & &  &  & & & 0 & 1 &\\
 & &  &  & & & -1 & 0 &\\ 
\end{bmatrix}
$$
\end{teorema}

De ahora en adelante, sea $\psi :\mathbb{R}^{2n} \times \mathbb{R}^{2n}\mapsto \mathbb{R}$ dada por 
$$ \begin{aligned}
\psi((x,y),(x',y'))=x.y'-y.x' ,
\end{aligned}$$
donde $x,y,x',y' \in \mathbb{R}^n$ con  $x.y$ el producto escalar usual de $\mathbb{R}^n$.
\end{frame}

\begin{frame}{Grupo de Heisenberg}
\begin{definicion}
Denotaremos con $\mathfrak{h}_n$ el álgebra de Heisenberg,definida sobre
$\mathbb{C}^n \times \mathbb{R}$  con el corchete de Lie dado por 
$$
[(v,t),(v',s)]=(0,\psi(v,v')),
$$
donde  $\psi (v,v')=Im(v.\overline{v'}).$
\end{definicion} 

\begin{definicion}
El grupo de Heisenberg es como variedad subyacente $\mathbb{C}^{n}\times \mathbb{R},$ con
ley de grupo dada por 
$$\begin{aligned}
(v,t)(v',s)=(v+v',t+s+ \frac{1}{2} \psi(v,v')),
\end{aligned}$$
según la fórmula de Baker-Campbell-Hausdorff.
\end{definicion}
\end{frame}


\begin{frame}{Grupo de Heisenberg}
\begin{definicion}
 Recordemos que, dados dos elementos $g,h$ en un grupo $G$, el conmutador de $g$ y $h$ es ${[}g,h{]}:=g h g^{-1} h^{-1}$.
 
 Llamaremos subgrupo conmutador $[G,G]$ al grupo generado por todos los conmutadores.
\end{definicion}
Notemos que, dados $g=(v,t),h=(v',s) \in \mathbb{H}_n$
$$
{[}g,h{]}=g h g^{-1} h^{-1}=(0,\psi(v,v'))
$$
\end{frame}

\begin{frame}{Automorfismos del grupo de Heisenberg}
 \begin{definicion}
 Recordemos que un automorfismo de álgebras de Lie $\alpha: \mathfrak{h}_n \mapsto \mathfrak{h}_n$ es un isomorfismo lineal tal que 
 $$\begin{aligned}
    \alpha({[}(v,t),(v',s){]})={[}\alpha(v,t),\alpha(v',s){]}
 \end{aligned}$$
\end{definicion}
\begin{definicion}
 Definimos el grupo simpléctico $Sp(2n,\mathbb{R})$ como ,
 $$
 Sp(2n,\mathbb{R})=\{ A \in GL(2n,\mathbb{R}): \psi(v,v')=\psi(Av,Av') \forall v,v' \in \mathbb{C}^n\}.
 $$
\end{definicion}

\end{frame}

\begin{frame}{Automorfismos del grupo de Heisenberg}
$$\alpha=
\begin{bmatrix}
A & c\\
b & a
\end{bmatrix},$$
con $A \in  \mathbb{R}^{2n} \times \mathbb{R}^{2n}$, $a \in \mathbb{R}$, $b,c \in \mathbb{R}^{2n}$.

\begin{itemize}
\item <2-> $c=0$
\item <2-> $a\neq0$
\item <2-> $a \psi(v,v')=\psi(A v,A v')$
\end{itemize}
\end{frame}

\begin{frame}{Automorfismos del grupo de Heisenberg}
Caso $a>0$:
 $$\alpha=
 \begin{bmatrix}
A & 0\\
b & a
\end{bmatrix}=
\delta_{a^{1/2}}
\begin{bmatrix}
S & 0\\
0 & 1
\end{bmatrix}
\begin{bmatrix}
I & 0\\
\tilde{b} & 1
\end{bmatrix}.
$$

Caso $a<0$:
$$\alpha=
 \begin{bmatrix}
A & 0\\
b & a
\end{bmatrix}=
\theta
\delta_{(-a)^{1/2}}
\begin{bmatrix}
S & 0\\
0 & 1
\end{bmatrix}
\begin{bmatrix}
I & 0\\
\tilde{b} & 1
\end{bmatrix}.
$$

Con, 
\begin{itemize}
\item $\delta_a (v,t) :=(a v,a^2 t)$
\item $\theta(v,t):=(\overline{v},-t)$
\end{itemize}
\end{frame}

poner antisimetrica

 \begin{frame}{Referencias}
  \begin{thebibliography}{1}
  \bibitem[Fo]{Fo} Folland G.  \emph{``Compact Heisenberg Manifolds as CR Manifolds''},The Journal of Geometric Analysis, Vol. 14 N. 3, 2004.(521-532)
  \bibitem[Ho]{Ho} Hoffman K.,Kunze R.  \emph{``Linear Algebra''},Prentice-Hall, 1973  
  \bibitem[Hu]{Hu} Hungerford T. W.  \emph{``Algebra''},Graduate Text in Mathematics, V. 73,Springer, 2003
  \bibitem[La]{La} Lang S.  \emph{``Algebra''},3. ed., Springer, 2004
  \bibitem[Th]{Th} Thangavelu S.  \emph{``Harmonic analysis on Heisenberg nilmanifolds''}.Revista de la Unión Matemática Argentina,V. 50, N. 2 , 2009. (75-93)
  \bibitem[To]{To} Tolimieri R.   \emph{``Heisenberg manifolds and Theta functions''}, Trans. of the A.M.S Vol 239, 1978. (293-319)  
  \end{thebibliography}
\end{frame}



\end{document}
